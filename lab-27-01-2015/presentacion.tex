\documentclass{beamer}
\usepackage[utf8]{inputenc}
\usepackage[spanish]{babel}
\usetheme{Goettingen}
\usecolortheme{default}
\useoutertheme{shadow}
\useinnertheme{rectangles}
\graphicspath{ {./figures/} }
\title[Laboratorio]{Introducción}
\subtitle{Herramientas de trabajo}
\author[Miguel]{Miguel Angel Piña Avelino}
\institute[UNAM]{
  Facultad de Ciencias, UNAM
}
\date{\today}

\begin{document}

\frame{\titlepage}

\begin{frame}
  \frametitle{Índice}
  \tableofcontents
\end{frame}

\section{Introducción}
\begin{frame}
  \frametitle{Introducción}
  En esta primera clase veremos las herramientas que utilizaremos en el curso. Las herramientas más importantes de las que haremos uso serán:
  \begin{itemize}
    \item JEE
    \item Netbeans
    \item GIT
    \item PostgreSQL
  \end{itemize}
\end{frame}
\section{JEE}
\begin{frame}[]
  \frametitle{¿Qué es JEE?}
  ¿Qué es JEE?
\end{frame}
\end{document}
