\documentclass{beamer}
\usepackage[utf8]{inputenc}
\usepackage[spanish]{babel}
\usetheme{Goettingen}
\usecolortheme{default}
\useoutertheme{shadow}
\useinnertheme{rectangles}
\graphicspath{ {./figures/} }
\title[Laboratorio]{Introducción}
\subtitle{Herramientas de trabajo}
\author[Miguel]{Miguel Angel Piña Avelino}
\institute[UNAM]{
  Facultad de Ciencias, UNAM
}
\date{\today}

\begin{document}

\frame{\titlepage}

\begin{frame}
  \frametitle{Índice}
  \tableofcontents
\end{frame}

\section{Introducción}
\begin{frame}
  \frametitle{Introducción}
  En esta primera clase veremos las herramientas que utilizaremos en el curso.
  Las herramientas más importantes de las que haremos uso serán:
  \begin{itemize}
    \item J2EE
    \item Netbeans
    \item GIT
    \item PostgreSQL
  \end{itemize}
\end{frame}
\section{J2EE}

\begin{frame}
  \frametitle{¿Qué es J2EE?}
  Java Platform, Enterprise Edition o Java EE (anteriormente conocido como Java
  2 Platform, Enterprise Edition o J2EE hasta la versión 1.4; traducido
  informalmente como Java Empresarial), es una plataforma de programación—parte
  de la Plataforma Java—para desarrollar y ejecutar software de aplicaciones en
  el lenguaje de programación Java.
\end{frame}

\begin{frame}
  \frametitle{¿Qué es J2EE?}
  \begin{itemize}
    \item Está orientado a empresas y a la integración entre sistemas.
    \item Incluye soporte a tecnologías para internet.
    \item Su base es J2SE.
  \end{itemize}
\end{frame}

\begin{frame}
  \frametitle{Características}
    Algunas de sus funcionalidades más importantes son:
  \begin{itemize}
    \item Acceso a base de datos (JDBC)
    \item Es utilizado por BEA, IBM, Oracle, Sun, y Apache Tomcat entre otros.
    \item Utilización de directorios distribuidos (JNDI)
    \item Acceso a métodos remotos (RMI/CORBA)
    \item Funciones de correo electrónico (JavaMail)
    \item Aplicaciones Web (JSP y Servlet)
    \item Uso de Beans, etc.
  \end{itemize}
\end{frame}

\section{Netbeans}

\begin{frame}
  \frametitle{¿Qué es Netbeans?}
  El NetBeans IDE es un IDE de código abierto escrito completamente en Java
  usando la plataforma NetBeans. El NetBeans IDE soporta el desarrollo de todos
  los tipos de aplicación Java (J2SE, web, EJB y aplicaciones móviles). Entre
  sus características se encuentra un sistema de proyectos basado en Ant,
  control de versiones y refactoring.
\end{frame}


\begin{frame}[fragile]
  \frametitle{Donde conseguir Netbeans}
  \begin{verbatim}
    https://netbeans.org/downloads/
  \end{verbatim}
\end{frame}

\section{GIT}

\begin{frame}
  \frametitle{¿Qué es GIT?}
  GIT es un software de control de versiones diseñado por Linus Torvalds,
  pensando en la eficiencia y la confiabilidad del mantenimiento de versiones de
  aplicaciones cuando estas tienen un gran número de archivos de código fuente.
\end{frame}

\begin{frame}[fragile]
  \frametitle{Tutorial de GIT}
  En internet hay varios tutoriales de GIT, pero este es uno de los que más me
  gustan:
  \begin{verbatim}
    https://try.github.io/levels/1/challenges/1
  \end{verbatim}
\end{frame}

\section{PostgreSQL}

\begin{frame}
  \frametitle{¿Qué es PostgreSQL?}
  PostgreSQL es un Sistema de gestión de bases de datos relacional orientado a
  objetos y libre, publicado bajo la licencia BSD.\\
  Actualmente se encuentra en la versión 9.3.5 y el release de la versión 9.4
  se publicó el 18 de diciembre del 2014.
\end{frame}

\begin{frame}[fragile]
  \frametitle{¿Dónde lo consigo?}
  \begin{verbatim}
    http://www.postgresql.org/
  \end{verbatim}
\end{frame}

\end{document}
